
\begin{document}

\chapter{Discussion}

\label{chapter:discussion}

This chapter discusses the results stated in Chapter \ref{chapter:results}, compare and contrast them with the Wei et al study \cite{wei_kinect_calibration}. For each of the primary tasks, namely Stationary, Steps, and Walk, the discussion will consist of how the average coordinates and joints distances change with different Kinect placements. The spread of distances values over different joint types in each task will also be discussed.

It is worth noting that Wei et al only studied Stationary and Steps tasks, with near parallel and 45$^{\circ}$ apart Kinects. Tthe current work will compare results with those in Wei et al's study where appropriate \cite{wei_kinect_calibration}.

\section{Stationary}
\label{sec:discussion_stationary}

This section discusses results in the Stationary task.

The Stationary task shows best results when the Kinects are parallel to each other and worst when they are 90$^{\circ}$ apart. All measures of distances follow the same trend, from $\Delta x$ to $\Delta z$ (See Figure \subref*{fig:stationary_coordinates}). The results show that coordinates distances in the Stationary task increases with increasing angle between Kinects.

In general, the $\Delta x$ values are the highest, then $\Delta z$ and $\Delta y$. This shows that the tracking algorithm makes the largest errors in the coordinates transformation of the x axis and least errors in the y axis. A feasible explanation would be that the heights of both Kinects are the same, and the participants are not moving on the y axis.

The standard deviations of $\Delta d$ in the average case is within the range of standard deviations across different Kinect placements. This shows that in the Stationary task the coordinates distances are consistent both within and between different Kinect placements.

The average joints distances are smallest for joints in the torso, namely the head, hips, knees, neck, shoulders, and spine (See Figure \subref*{fig:stationary_joints}). These joints appear to have similar coordinates distances. On the contrary, joints in the arm and foot regions have higher coordinates distances. The researcher speculates that the results are influenced by Kinect's internal performance; it produces more reliable and consistent results in the torso compared to other regions of the body.

All average coordinates distances are also higher for the left joints compared to right joints. In summary, right hand side joints in the torso have the smallest coordinates disances, whereas left hand side joints in the arm and foot regions have the highest coordinates distances. The researcher also speculates that these results are due to the Kinect itself, which produces higher quality depth map in the torso.

Wei et al reported lower values compared to the current study \cite{wei_kinect_calibration}. In their Stationary task (Average Difference before Movement) with Parallel ($4.25^{\circ}$) apart Kinects, the coordinates distances are $0.00$, $1.00$, and $2.00$ cm for $\Delta x$, $\Delta y$, and $\Delta z$, respectively. The $\Delta d$ will be $2.24$ cm, which is lower than the $3.52$ cm (See Table \ref{table:stationary_coordinates_values}) found in the current study. In their same task with $45^{\circ}$ ($44.37^{\circ}$) apart Kinects, the coordinates distances are $1.00$, $1.00$, and $1.50$ cm for $\Delta x$, $\Delta y$, and $\Delta z$, respectively. The $\Delta d$ will be $2.06$ cm, also lower than the $6.95$ cm found in the current study.

\section{Steps}
\label{sec:discussion_steps}

This section discusses results in the Steps task.

The coordinates distances in the Steps task are higher compared to those in the Stationary task for every type of Kinect placements. The increase in coordinates distances is anticipated, because the task requires the participants to walk around in the environment, causing them to move closer or further away from the cameras. The movements will lead to increased liklihood of the tracking algorithm making more errors compared to when the participants are standing still. The following section will comment and explain the similarities and differences in the results between the Steps and Stationary tasks.

The Steps task also shows best results when the Kinects are parallel to each other and worst when they are 90$^{\circ}$ apart. All measures of distances follow the same trend, from $\Delta x$ to $\Delta z$ (See Figure \subref*{fig:steps_coordinates}). Alike to the results in the Stationary task, the studies for the Steps task show that coordinates distances in also increase with increasing angle between Kinects.

For all Kinect placements, the $\Delta x$ values are the highest, then $\Delta z$ and $\Delta y$. The results support the findings in the Stationary task. This shows that the tracking algorithm produces consistent coordinates transformation across different tasks.

The standard deviation of $\Delta d$ within each type of Kinect placement is lower than the same setup in the Stationary task. The standard deviations of $\Delta d$ in the Stationary task are $1.33$, $2.67$, $4.45$, and $3.95$ cm, for Parallel, 45$^{\circ}$, and 90$^{\circ}$ apart Kinects, respectively. On the other hand, the Steps task yields $0.90$, $1.92$, $3.46$, and $9.32$ for the same measures. These values suggest that the algorithm produces less variations within different Kinect configurations in the Steps task compared to the Stationary task. However, the Steps task has a higher standard deviation of $\Delta d$ in the average case, compared to that of in the Stationary task. Even though the system shows smaller variations within different Kinect placements in the Steps task, it produces large variations between different conditions of Kinect placements. The system copes better to changes in Kinect placements in the Stationary task than in the Steps task.

There is a less visible difference in the coordinates distances between joints in the torso and other body regions (See Figure \subref*{fig:steps_joints}), but it remains when examined closely. However, the pattern that left joints have higher coordinates distances is still very noticeable. Overall, the variations of the distances in each joint are still large, except for the $\Delta y$ distances. The $\Delta y$ distances and their standard deviations reamin small.

Wei et al also reported lower results compared to the current study. In their Steps task (Average Difference after Movement) with Parallel ($4.25^{\circ}$ apart) Kinects, the coordinates distances are $2.00$, $1.28$, and $3.78$ cm for $\Delta x$, $\Delta y$, and $\Delta z$, respectively. The $\Delta d$ will equal to $4.46$ cm, which is lower than the $6.87$ cm (See Table \ref{table:steps_coordinates_values}) found in the current study. In their same task with $45^{\circ}$ ($44.37^{\circ}$) apart Kinects, the coordinates distances are $4.28$, $1.64$, and $5.28$ cm for $\Delta x$, $\Delta y$, and $\Delta z$, respectively. The $\Delta d$ will equal to $6.99$ cm, also lower than the $12.80$ cm found in the current study.

\section{Walk}
\label{sec:discussion_walk}

This section discusses results in the Walk task.

The coordinates distances in the Walk task are higher compared to those in the Stationary and Steps tasks for every type of Kinect placements. The increase in coordinates distances in the Walk task has the same cause as previously mentioned in Section \ref{sec:discussion_steps} for the Steps task. Becuase walking movements are even larger than the stepping and stationary movements, the results in the Walk task will be higher compared to the other two tasks.

Similar to the previous two tasks, the results in the Walk task show smaller coordinates distances with parallel Kinects, and increasing distances with larger angles (See Figure \subref*{fig:walk_coordinates}). Likewise, the $\Delta x$ values are still the highest, follwed by $\Delta z$ and $\Delta y$. $\Delta y$ is still nearly invariant to changes in Kinect placement (See Figure~\ref{fig:walk_joints}). Its standard deviation is the lowest compared to the standard deviations of all other distances measures, in all the tasks seen so far (Stationary, Steps, and Walk) with all different Kinect placements (Parallel, $45^{\circ}$, and $90^{\circ}$ apart Kinects).

The average and standard deviation of $\Delta y$ is the only value that hardly changes across different tasks and Kinect placements. In the Stationary task, the average $\Delta y$ is $3.07$ cm, with a standard deviation of $1.60$ cm. In the Walk task, the average $\Delta y$ only increases to $4.81$ cm, with a standard deviation of $1.42$ cm. The average $\Delta y$ in the Steps task is in between these values. The results can be found in Table~\ref{table:stationary_coordinates_values}, Table~\ref{table:steps_coordinates_values}, and Table~\ref{table:walk_coordinates_values}. These support the argument that $\Delta y$ is steady throughout the tracking process, regardless of tasks and Kinect placements.

In both the Steps and Walk tasks, the standard deviation of coordinates distances within each Kinect placement is small compared to the same value in the average case. The same relationship is not observable in the Stationary task. The researcher argues that increasing angle between multiple Kinects have a larger impact in non-stationary tasks.

There are common patterns in the joints distances between different tasks. Firstly, all three tasks show higher distances values for the left joints. Furthermore, there is a difference between joints in the torso and other boy regions, to varying degrees across tasks. In additional, all three tasks show that $\Delta x$ and $\Delta z$ are closer to each other than to $\Delta y$.

Wei et al did not run experiments containing longer, continuous walking tasks. The researcher has not found results for a similar task in the literature.

\section{Stationary, Steps, Walk}
\label{sec:discussion_scenarios}

The tasks Stationary, Steps, and Walk are ordered in increasing complexity, where the former requires zero movements, and the latter requires constant movements. The studies show that coordinates distances increase with increasing complexity of the tasks (See Figure \ref{fig:results_coordinates_kinect_all}). The relationship can be attributed to increasing amount of joints movements and turning of the shoulder, of which the coordinates transformation uses for calculating the angle between the Kinects and the person. In addition, when averaged over different Kinect placements, the coordinates distances for different joints in Stationary, Steps, and Walk tasks are roughly the same, but there is a distinct difference between coordinates of the left and right joints (See Figure~\ref{fig:results_joints_kinect_all}).

The Kinect placements of parallel, 45$^{\circ}$ and 90$^{\circ}$ are ordered in increasing angle. The studies show that coordinates distances increase with increasing angle (See Figure~\ref{fig:results_coordinates_task_all}). The angle between Kinects correlates to the degree of rotation used in the transformation of multiple skeletons. The larger the angle is between Kinects, the skeletons will be rotated more, hence producing larger coordinates differences. The joints distances follow a similar pattern as shown in the plot with different tasks averaged over different Kinect placements. The left joints also have larger coordinates differences compared to the right joints, but there is less variation between joints, and the averages are slightly smaller (See Figure~\ref{fig:results_joints_task_all}).

When varying only either the complexity of the tasks or the angle of Kinect placement, the results will show similar trends. In short, the more complex the task or the larger the angle between multiple Kinects, the worse results will be. The scenario where $\Delta d$ is the smallest is the Stationary task with parallel Kinects ($3.52$ cm), and the scenario where $\Delta d$ is the highest is the Walk task with 90$^{\circ}$ apart Kinects ($32.38$ cm). However, the average cases of varying only the task or Kinect placement show promising results, $14.10$ cm and $10.57$, respectively.

The worse average cases show that the $\Delta d$ distances are around $20$ cm (See Figure~\ref{fig:results_three_coordinates_joints}). This boundary is still within the personal space, or the space where only one person is most likely to exist. The results show preliminary success in tracking people using transformed 3D coordinates to find their spatial positions and joints.

\textbf{Time}

\section{Obstacle}
\label{sec:discussion_obstacle}

\textbf{TODO}

\section{Interaction}
\label{sec:discussion_interaction}

\textbf{TODO}

\section{Summary}

The researcher summarizes the findings as follows:

\begin{enumerate}
  \item $\Delta x$, $\Delta y$, $\Delta z$, and $\Delta d$ increase with increasing complexity of tasks, from being stationary to walking
  \item $\Delta x$, $\Delta y$, $\Delta z$, and $\Delta d$ increase with increasing angle between Kinects
  \item The torso (head, hip, knee, neck, shoulder, and spine) is more reliable than other body regions
  \item The left joints are more reliable than the right joints
  \item $\Delta d$ is less than $15$ cm on average, over different tasks and Kinect placements
\end{enumerate}

\section{Criticisms}
\label{sec:discussion_criticisms}

The data collection (logging) procedure was flawed due to software failure. The logger did not log the same length of stationary movement for all participants and different Kinect placements. The lengths of the results in the Stationary task do not all equal to ten seconds as described in Chapter~\ref{chapter:studies}.

The data cleaning process is also not rigourous. In both the Steps and Walk tasks, the participants remained in the same positions between instructions. The skeletons data for when the participants remained in the same positions should be discarded before doing any further analysis investigating the effects of task complexity and Kinect placements.

The analysis did not include significance testings on the hypotheses. The results are only descriptive, not inferrential.

\section{Future Work}
\label{sec:discussion_future_Work}

Future work can be divided into two domains, one for the tracking system, including the algorithms and application, and the other one for the user studies.

\end{document}
