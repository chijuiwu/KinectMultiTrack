
\begin{document}

\chapter{Studies}

\label{chapter:studies}

\section{Motivation}
\label{sec:motivation}

A series of user studies are designed to evaluate the system's accuracy at tracking people in different scenarios. The accuracy of the tracking algorithm, or essentially the coordinate transformation algorithm, is measured by the differences in the joint coordinate between multiple potential skeletons of the same person. The studies will require participants to move around in front of multiple Kinects alone and with other participants. The software will log participants' positions from tracking, and these data will provide a quantitative feedback on the accuracy of the algorithm in different Kinect configurations and user scenarios.

To reiterate, a potential skeleton is a skeleton from a single Kinect field of view. One person may have multiple potential skeletons when they are visible to many Kinects. The application is most useful for its ability to transform any potential skeleton into any Kinect's camera space. The potential skeleton in the current Kinect field of view would be unaffected, but the other potential skeletons that were in other Kinects fields of view would have slight deviations in their joint coordinates. The user studies attempt to capture such deviations in all possible cases.

\section{Hypotheses}
\label{sec:hypotheses}


\end{document}
