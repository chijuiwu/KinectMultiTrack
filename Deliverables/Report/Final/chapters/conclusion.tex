
\begin{document}

\chapter{Conclusion}

\label{chapter:conclusion}

The current system implements a people tracking algorithm using multiple Kinects and 3D coordinate system transformation based on unit quaternions. It merges multiple skeletons from different Kinects field of view into a new coordinate system, then transformed to the user-selected perspective of one of the connected camera. The system can resolve one camera occlusion by using the depth sensor information from the other available Kinect when users are obstructed by objects in the scene or during human interactions. The researcher carried out a series of user studies investigating the accuracy of the current methodology. The studies measure the coordinates distances between multiple skeletons of the same person from different Kinects field of view after coordinate transformation, expressed in terms of $\Delta x$, $\Delta y$, $\Delta z$, and $\Delta d$ in centimeters. The values are averaged over all joint types provided by the Kinect. The studies consist of three different tasks with increasing complexity and three different Kinect placements with increasing angle. Results show that average coordinates distances increase with both the complexity of the task and the angle between Kinects. Even though the studies show lower accuracy compared to the previous work, the average coordinate distance over different scenarios is still within the region of person space. This finding allows opportunities for integrating algorithms which exploit people's spatial positions into the current system. In the future, a tracking system would monitor the blood oxygenation level in patients in a occluded hospital setting. A context-aware user interface can show users information based on what they can and cannot see from their current position. Mobile robots can share spatial information to allow an agent to approach a person safely. Tracking people through occlusion can enable interactive systems to learn from once hidden information and deliver purposeful actions.

\end{document}
