
\begin{document}

\chapter{Design}

\label{chapter:design}

\section{Requirements}
\label{sec:design_requirements}

The application should be intuitive and easy to use. Since it is a prototype demonstrating the capability of the tracking system, it puts large emphasis on the skeleton visualization, showing that the combined skeletons match the expected outcome of the tracking process. The application displays the skeletons before and apply applying coordinate transformation and skeleton matching. The combined skeletons should render at the same speed as the server receives BodyFrames from multiple sources. 

The application provides the end users essential functionalities for running the application, including to start and stop the server, recalibrate, view tracked skeletons from different views, and send the average skeleton stream to other applications. The logger should also store the tracking data on demand.

The researcher has discarded the following requirements for the software, due to time constraints and the scope of the project:

\begin{itemize}
  \item The security of the application.
  \item The privacy of users' tracking information
  \item The scalability and robustness of the client and server.
\end{itemize}

\section{Software stack}
\label{sec:design_software_stack}

The current work uses Kinects for XBox One. The Microsoft Kinect SDKs are in both C++ and C\#. The system is written in C\# with the 4.5 .NET framework, and the user interface is created using the .NET WPF framework. C\# is chosen over C++ because of the ease of development given the researcher's background in Java, which is considered to be a similar language to C\#. The WPF framework is selected over traditional Windows Forms, because the the latest examples included in the official SDK use this framework.

\section{System architecture}
\label{sec:design_architecture}

The main components of the system consist of a number of clients, a server, a tracker, a user interface, and a logger (See Figure~\ref{fig:system_architecture}). The server passes on the Kinect BodyFrames received from the clients to the tracker. The tracker then processes the data and returns the result to the server. The server redirects the result to the user interface, which then displays the tracking result as a skeleton visualization. When required by the end user, the the server will signal the logger to write tracking result to files.

The system topology consists of one or more machines in a client-server model. The latest Kinect v2 SDK at the time of writing (version 2.0.1410.19000) still does not support running multiple Kinects on a single machine, as a result, the system leverages the TCP/IP protocol for communicating between multiple Kinects. In the current system architecture, each client is running one Kinect. There is only one server, and any client machine can also run the server. All clients send Kinect Body frames to the server. The server is the workhorse of the system. It serves incoming client connections, establishes network streams with the clients, runs the user interface and exchanges information with the track which runs the tracking algorithm, and lastly, informs the logger to write tracking data to files when required.

\begin{figure}[!h]
  \centering

  \includegraphics[width=0.8\linewidth]{figs/system_architecture}
  
  \caption{The system architecture. Clients running the Kinects will send serialized skeleton data to the server. The server oversees the tracking algorithm in the server and the skeleton visualization in the user interface. The server sends raw skeleton data to the tracker. The tracker returns a new tracking outcome by updating the initial detection result. The server invalidates the user interface and requests a new skeleton visualization with the most recent tracking result.}
  
  \label{fig:system_architecture}
\end{figure}

\end{document}
