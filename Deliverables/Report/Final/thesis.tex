\documentclass{report}

%\documentclass[journal]{IEEEtran}
%\documentclass{report}
%\documentclass{acta}

\usepackage{geometry}

\geometry{
 a4paper,
 total={210mm,297mm},
 left=25mm,
 right=25mm,
 top=25mm,
 bottom=25mm,
 }

\newcommand{\HRule}{\rule{\linewidth}{0.5mm}}

\setlength{\parskip}{1em}

\usepackage{amsmath, amssymb}

\usepackage[hyphens]{url}

\usepackage[pdftex]{graphicx}

\usepackage{float}

\usepackage{relsize}

\usepackage{algorithm}

\usepackage{fancyvrb}

\usepackage[noend]{algorithmic}

\usepackage{subfiles}

\usepackage{titlecaps}

\usepackage{fancyhdr}

\usepackage{amsthm}

\usepackage{subfig}

\usepackage{caption}

\usepackage[noabbrev]{cleveref}

\usepackage{bm}

\usepackage{tabulary}

\captionsetup[subfigure]{subrefformat=simple,labelformat=simple,listofformat=subsimple}
\renewcommand\thesubfigure{(\alph{subfigure})}

\DeclareSymbolFont{boldoperators}{OT1}{cmr}{bx}{n}
\SetSymbolFont{boldoperators}{bold}{OT1}{cmr}{bx}{n}
\edef\bar{\unexpanded{\protect\mathaccentV{bar}}\number\symboldoperators16}

\renewcommand{\algorithmicrequire}{\textbf{Input:}}
\renewcommand{\algorithmicensure}{\textbf{Output:}}

\newcommand{\tab}[1]{\hspace{.2\textwidth}\rlap{#1}}
\newcommand{\itab}[1]{\hspace{0em}\rlap{#1}}

\renewcommand{\thefootnote}{\fnsymbol{footnote}}
\newcommand{\Acronym}[1]{\ensuremath{{{\texttt{#1}}}}}
\newcommand{\Symbol}[1]{\ensuremath{\mathcal{#1}}}
\newcommand{\Function}[1]{\ensuremath{{ \textsc{#1}}}}
\newcommand{\Constant}[1]{\ensuremath{{\texttt{#1}}}}
\newcommand{\Var}[1]{\ensuremath{{{\textsl{#1}}}}}
\newcommand{\False}{\Constant{false}}
\newcommand{\True}{\Constant{true}}
\newcommand{\Null}{\Constant{null}}

\newcommand{\dif}{\ensuremath{{\mathrm{d}}}}

\newcommand{\Name}{\Acronym{Dodger}}

\newcommand{\Revision}[1]{\textcolor{red}{#1}}
\newcommand{\R}{\ensuremath{\mathbb{R}}}
\newcommand{\Traj}{\ensuremath{\zeta}}
\newcommand{\Tree}{\Symbol{T}}
\newcommand{\pair}[1]{\ensuremath{\langle#1\rangle}}

\newcommand{\argmin}[1]{\underset{#1}{\operatorname{arg}\,\operatorname{min}}\;}
\newcommand{\argmax}[1]{\underset{#1}{\operatorname{arg}\,\operatorname{max}}\;}

\newcommand{\mychange}[1]{\textcolor{red}{#1}}

\newtheorem{theorem}{Theorem}

\newtheorem{lemma}{Lemma}

\newtheorem{corollary}{Corollary}

\pagestyle{fancy}

\fancyhf{}

\fancyhead[CE,LO]{\leftmark}

\fancyfoot[LE,CO]{\thepage}

\renewcommand{\headrulewidth}{0.5pt}

\usepackage[Sonny]{fncychap}

\begin{document}

\sloppy

\subfile{formal/title}

\pagebreak

\begin{abstract}

The current work is about tracking people with multiple Kinects. The goal is to reliably track people in uncertain, occluded real-world environments. The final submission contains an interactive application that demonstrates the tracking system as well as a series of user studies reporting the success of the system. The strengths and limitations of the system are discussed. An outline of future work to make the current system deployable in the real life is described.

\end{abstract}

\pagebreak

\subfile{formal/declaration}

\pagebreak

\tableofcontents

\listoffigures

\subfile{chapters/introduction}

\subfile{chapters/background}

\subfile{chapters/objectives}

\subfile{chapters/current_approach}

\subfile{chapters/design}

\subfile{chapters/implementation}

\subfile{chapters/testing}

\subfile{chapters/studies}

\subfile{chapters/results}

\subfile{chapters/discussion}

\subfile{chapters/conclusion}

\subfile{chapters/notes}

\subfile{chapters/ethics}

\subfile{chapters/acknowledgements}

\subfile{chapters/appendix}

\bibliographystyle{ieeetr}

\bibliography{thesis}

\end{document}
