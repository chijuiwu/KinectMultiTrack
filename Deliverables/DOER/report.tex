\documentclass[paper=a4, fontsize=11pt]{scrartcl}
\usepackage[T1]{fontenc}
\usepackage{fourier}

\usepackage[english]{babel}
\usepackage[protrusion=true,expansion=true]{microtype}	
\usepackage{amsmath,amsfonts,amsthm} % Math packages
\usepackage[pdftex]{graphicx}	
\usepackage{url}
\usepackage{hyperref}
\usepackage{apacite}
\usepackage{enumitem}
\usepackage[margin=0.75in]{geometry}

\usepackage{sectsty}
\allsectionsfont{\centering \normalfont\scshape}

\usepackage{fancyhdr}
\pagestyle{fancyplain}
\fancyhead{}
\fancyfoot[L]{}
\fancyfoot[C]{}
\fancyfoot[R]{\thepage}
\renewcommand{\headrulewidth}{0pt}
\renewcommand{\footrulewidth}{0pt}
\setlength{\headheight}{10pt}
\setlength{\headsep}{10pt}
\setlength{\textheight}{700pt}
\setlength{\footskip}{50pt}

\numberwithin{equation}{section}
\numberwithin{figure}{section}
\numberwithin{table}{section}

\newcommand{\horrule}[1]{\rule{\linewidth}{#1}}

\title{
		\vspace{-3ex}
		\usefont{OT1}{bch}{b}{n}
		\normalfont \normalsize \textsc{School of Computer Science, University of St Andrews} \\ [25pt]
		\horrule{0.5pt} \\[0.2cm]
		\huge Tracking People using Multiple Kinects: Description, Objectives, Ethics, Resources \\
		\horrule{2pt} \\[0.2cm]
		\vspace{-2ex}
}
\author{
		\normalfont \normalsize
        Chi-Jui Wu\\[-3pt] \normalsize
        {cjw21@st-andrews.ac.uk}\\[-3pt] \normalsize
        \today
}
\date{}

\begin{document}
\maketitle

\section{Description}

Real time detecting and tracking multiple people have been achieved in systems that use RGB-D data\cite{track_rgbd}, shape analysis, stereo and intensity images\cite{w4s}, and appearance-based tracking\cite{track_robust}, just to name a few. Kinect is an inexpensive sensor that produces depth, RGB, infrared, and audio streams at high frame rates. The current work will build on top of Kinect's existing skeleton detection and tracking system.

The goal of the project is to develop and evaluate an algorithm that uses multiple Kinects to track moving people in real-time. Using multiple Kinects has the advantage of a larger field of view, potentially from different angles, used to capture more information about the people in sight. On the other hand, necessary Kinect calibration and data synchronization will introduce additional complexity. The tracking algorithm should be accurate and robust. It should also resolve the common problem of occlusion in detection and tracking.

Some applications of tracking multiple people include surveillance\cite{surveillance}, tracking with mobile robots\cite{mobile}, facial animation of digital avatars\cite{facial_animation}, and the possibility of monitoring their blood oxygenation levels.

\section{Objectives}

The objectives are ordered by the difficulty of the tasks. They should be completed using two Kinects, unless explicitly stated otherwise.

\begin{enumerate}[topsep=0pt,itemsep=0ex,partopsep=1ex,parsep=1ex]
\item Literature review
	\begin{itemize}[topsep=0pt,itemsep=0ex,partopsep=1ex,parsep=1ex]
		\item Identify the state of the art techniques in people tracking using RGB, depth, and infrared data
		\item Identify the limitations of the current approaches
		\item Identify important algorithms that can be incorporated to the current work.
	\end{itemize}
\item Calibrate the Kinects using a test target
\item Synchronize the Kinects input from two different machines (The current SDK may not allow one machine to use two Kinect v2 at once)
	\begin{itemize}[topsep=0pt,itemsep=0ex,partopsep=1ex,parsep=1ex]
		\item Merge the RGB, depth and infrared data for each person detected.
	\end{itemize}
\item Recognize a single user (Match the skeletons from different Kinects)
\item Track a single user (Extend the Kinect's field of view)
\item Track two users (Specify the positions of the Kinects)
\item Evaluate the current system (Also applies to further objectives)
\item Automate the calibration process without using a test target
\item Track six users reliably in constraint test environment
\item Track six users who have various poses and interactions
\item Track six users in different indoor environments
	\begin{itemize}[topsep=0pt,itemsep=0ex,partopsep=1ex,parsep=1ex]
		\item brightness
		\item various occlusion objects (Partial and full occlusion)
		\item simulated physical office and home space
	\end{itemize}
\item Track six users in outdoor environments
\end{enumerate}

\section{Ethics}

Kinect is a commercial off-the-shelf hardware that has been approved for use worldwide. To help the researcher develop and verify the software, the project will require many recordings of one or more moving people. There is a small risk that people will exhibit inappropriate behaviour towards one another during the recording sessions. In such cases, the session will be immediately terminated. Participants will be required to sign a consent form, and they may leave the recording session at anytime without giving any reason. The recordings will be stored in an anonymous format and only be accessible to the researcher. The researcher will also incentivize the participants with sweets.

\section{Resources}

\begin{itemize}[topsep=0pt,itemsep=0ex,partopsep=1ex,parsep=1ex]
	\item Two Kinect for Windows v2
		\begin{itemize}[topsep=0pt,itemsep=0ex,partopsep=1ex,parsep=1ex]
			\item Note: May require additional Kinects and computers with 3.0 USB ports
		\end{itemize}
	\item A large space to accommodate at least six people
		\begin{itemize}[topsep=0pt,itemsep=0ex,partopsep=1ex,parsep=1ex]
			\item To make recordings of moving people, for development, testing, and demonstration purposes
		\end{itemize}
\end{itemize}

*The supervisor will provide all of the items above.

\bibliographystyle{apacite}
\bibliography{report}

\end{document}
